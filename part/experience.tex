\section{Experience Sharing}

For the carts part, we mainly focus on the choice of motors and servo. At first, we want to use advanced 130 motor to drive the cart in order to make the speed fast. But the power needed is too much and our motor driver can not afford. So we change to the 180 motor. However, it is heavier and even less powerful. So finally, we choose the N20 DC motor. Though we sacrifice the speed, the torsion becomes bigger. In addition, when we apply the wheels with bigger radius, the speed actually does not decrease. So the N20 motor is more suitable in driving the cart. 
\par
In terms of the servo, it is used to lift up the cup. We choose servo instead of motor because assembling servo is much easier than motor. We need another motor driver if the motor is used to lift the cup. But for servo, we don’t have this problem. Also, the torsion the servo can give is also much bigger. Considering the steering wheel has a large radius
$$M=F\times r$$
So when the radius becomes bigger, say ten times. $\frac{dr}{dt}$, which is proportional to the radius would also become ten times. 
\par
Actually, the speed of motor would be greatly influenced due to the torsion. However, when it come to the servo, it works quite well.