\section{Experience Sharing}

\subsection{Bridge}
In the beginning, our group produce paper beams with square cross section to make the first and the second bridge. However, it turns out that this structure is not stable enough to bear the weight of our cart. Then we change our paper from 70 grams per square meter to 80 grams per square meter. According to Professor Shen, a paper beam with rectangular cross section behaves better in carrying a heavy thing, so we decide to give up the previous design and buy new moulds. We complete making our third and fourth bridges before the Game Day. Thanks to the advanced structure, it behaves well on the final stage and surpasses all the other groups.
\par
During the process in making paper bridge, we improve our abilities to use fundamental structure to solve realistic problems and to optimize solutions. Another valuable experience we gain from project one is cooperation and teamwork, especially in producing the paper bridge. We separate the time-consuming task into several parts and assign them to each member. In this way, we manage to produce one bridge after another in high efficiency.
\par
The weight is a problem of our final bridge. It weighs 39.7g. Although the weight still receives full score, a much lighter one can have the same function. We are not able to make more bridges due to the limited time. But for future improvement, we will remake paper beams with two-layer structure and cut down the length of paper pillars.

\newpage

\subsection{Carts}
For the carts part, we mainly focus on the choice of motors and servo. At first, we want to use advanced 130 motor to drive the cart in order to make the speed fast. But the power needed is too much and our motor driver can not afford. So we change to the 180 motor. However, it is heavier and even less powerful. So finally, we choose the N20 DC motor. Though we sacrifice the speed, the torsion becomes bigger. In addition, when we apply the wheels with bigger radius, the speed actually does not decrease. So the N20 motor is more suitable in driving the cart.
\par
In terms of the servo, it is used to lift up the cup. We choose servo instead of motor because assembling servo is much easier than motor. We need another motor driver if the motor is used to lift the cup. But for servo, we don’t have this problem. Also, the torsion the servo can give is also much bigger. Considering the steering wheel has a large radius
$$M=F\times r$$
So when the radius becomes bigger, say ten times. $\frac{dr}{dt}$, which is proportional to the radius would also become ten times. 
\par
Actually, the speed of motor would be greatly influenced due to the torsion. However, when it come to the servo, it works quite well.
