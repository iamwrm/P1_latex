\section{Design Overview}
\subsection{Bridge}
\subsection{Cart}

The cart has ten main parts to fulfill the requirements: base, servo, wheel,
motor, battery, Arduino Board, Motor Driver and voltage transformer.

\bigskip
\textbf{Base: Carbon Fibre} \\
\indent
Although we considered using Acrylic board, the fact that it is too thick, too
heavyand easy to break make us decide to use carbon fibre.
Carbon fibre is much lighter and stronger.
It also has a better looking. 

\bigskip
\textbf{Servo: MG996R}  \\
\indent
We decided to use servo instead of motor to lift water because servo is more
powerful and easy to assemble. 
The first type we used is P0090.
It was not powerful enough.
So we finally choose MG996R. 

\bigskip
\textbf{Wheel: 3D-printed Plastic Wheel } \\
\indent
In order to fit the bridge, and meanwhile lower the influence to other parts of
the cart, we decided to print our own wheels in 3D.
The specially made wheels match the bridge well and guide the cart’s direction.

\bigskip
\textbf{Motor:  gear motor N20 } \\
\indent
At first, we believe the high speed motor like 130 and 180 would give our cart
extraordinary speed.
But they were not powerful enough to move the cart on the bridge.
Also, the high speed motor has difficulty while stopping.
So we finally choose N20 gear motor.
It becomes slower but more powerful and stable.
In addition, N20 is also lighter than any high speed motor.  

\bigskip
\textbf{Battery: 11.1V Battery } \\
\indent
The battery we chose has voltage 11.1V, which can well charge the Arduino board,
the motor board and the servo.
We choose 11.1V because we can have more choices of voltage using transformer.
We can adjust the speed of motor and servo eassily.  

\bigskip
\textbf{Arduino Board: UNO } \\
\indent
We first used Arduino UNO which was very light and small; however, we found it
too unreliable and not proper enough to be assemble on the cart, so we changed
to Arduino UNO.
It can provide all the functions we need stalely with a little weight increase.

\bigskip
\textbf{Motor Driver: Motor Driver L9110 } \\
\indent
We first used Motor Driver L298n, which was provided by our project instructor;
however, we found the cooling n too heavy and to a certain degree useless for
our low-power cart, so we changed to Motor Driver L9110.

\bigskip
\textbf{Voltage transformer:} \\
\indent
First, we did not use transformer to power the servo but Arduino board.
It takes more than 5 seconds to pull up the water.
After taking some calculation, the increase in voltage will shorten the time to
70\%.
So we finally applied transformer and lower the 11.1V from battery to 7.4V to
power the servo.

\subsection{Overall}